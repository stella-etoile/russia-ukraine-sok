\abstract{
Since the beginning of the full-scale Russian invasion of Ukraine in February 2022, both Russia and Ukraine have enacted extensive online censorship practices and experienced attacks on internet infrastructure. Existing research falls into two categories: studies capturing censorship events immediately after the escalation of the war, and studies documenting ongoing or more recent incidents. In this SoK, we synthesize four years of censorship-related findings across technical, political, and social domains. By consolidating disparate measurement studies, legal analyses, and examinations of civilian harms, we provide an integrated understanding of how information control has evolved throughout the conflict. Our goal is to offer a comprehensive, up-to-date overview that can inform future research, help evaluate defensive strategies, and deepen understanding of information warfare in modern conflicts.
}
