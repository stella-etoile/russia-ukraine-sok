\abstract{
Russia's full-scale invasion of Ukraine in February 2022 initiated a long period of censorship attacks affecting civilian populations and critical infrastructure. This systematization of knowledge summarizes research across technical measurements, socio-political analyses, and legal perspectives to provide a comprehensive understanding of how internet censorship has been used throughout the conflict, examining techniques including DDoS attacks, DNS and BGP manipulation, TCP/HTTP interference, TLS handshake delays, satellite attacks, and wiper malware deployments. Our analysis reveals that while Russia initially attacked Ukrainian infrastructure, it has since focused predominantly on isolating itself from the global internet and censoring its own citizens, with evidence showing intensifying restrictions through 2025. We also identify a significant lack of technical studies on recent developments and insufficient attention to domestic censorship as a war tactic.
}
