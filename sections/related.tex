\section{Related Work}\label{sec:related}

Existing research on censorship in the Russo--Ukrainian War falls into three main thematic areas: (1) general overviews of censorship and information control, (2) technical and mea-\\surement-based studies, and (3) socio-political analyses of legal and human-rights implications.

General overviews include discussions of propaganda, information control, and the legal frameworks that govern speech restrictions. For example, Kaye examines censorship, surveillance, and state pressure in Russia and evaluates responses by Ukraine, the EU, and technology companies, highlighting tensions between wartime measures and international human rights obligations.\cite{kaye2022online} Koltay similarly analyzes censorship as a tool against state disinformation and the implications for freedom of expression in the context of the Russian--Ukrainian War.\cite{koltay2023censorship}

Technical and measurement-based studies analyze censorship mechanisms, blocking behavior, cyberattacks, and network disruptions. Singh and Acharya evaluate whether Russian infrastructure influences blocking experienced by Ukrainian users and investigate the role of Starlink in maintaining connectivity.\cite{singhcyberspace} Ramesh et al.\ document large-scale network responses to Russia's 2022 invasion, including routing changes, outages, and filtering behavior that affect internet freedom.\cite{ramesh2023network} Eichensehr surveys cyber operations during the invasion---including the Viasat satellite attack and limited but targeted wiper and DDoS campaigns---and reflects on their implications for international law.\cite{Eichensehr_2022} Knockel et al.\ focus on in-platform censorship on VKontakte, analyzing moderation practices and content control within a major Russian social network.\cite{knockel2023not}

Socio-political analyses and human-centered studies explore the lived effects of censorship and cyber harm. Kulyk et al.\ document how network outages, censorship, data leaks, impersonation, and disruption of online services produce both primary and secondary harms for civilians, ranging from information vacuums and communication loss to physical danger and psychological distress.\cite{10.1145/3706599.3719906} Tavakkoli et al.\ study target preferences of pro-Russian and pro-Ukrainian groups, showing that news and media, government, business, and financial services are repeatedly attacked to disrupt information flows, public interaction, and economic stability.\cite{Tavakkoli2025Frontlines}

Our work extends these literatures by synthesizing them into a unified overview that spans the full timeline of the conflict and integrates both technical findings and their socio-political implications. Rather than focusing on a single measurement campaign, legal framework, or attack family, we provide a structured map of censorship practices and consequences in the ongoing Russia--Ukraine conflict as it unfolds on the internet.
