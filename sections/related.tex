\section{Related Work}\label{sec:related}

Existing research on censorship in the Russo--Ukrainian War falls into three main thematic categories: (1) general overviews of censorship and information control, (2) technical and measurement-based studies, and (3) socio-political ramifications of censorship and cyber operations.

\subsection*{General Overview}
Scholars examining broad censorship and information-control practices highlight the legal, political, and institutional pressures shaping online speech during the conflict. Kaye provides an overview of Russian censorship, propaganda, surveillance, and the legal restrictions imposed on online platforms, as well as responses by Ukraine and the EU\cite{kaye2022online}. He argues that while states rely on frameworks such as the ICCPR and ECHR, wartime information-control measures raise tensions between national security and freedom of expression. Koltay similarly analyzes censorship as a tool against state disinformation, exploring the implications of speech restrictions and uneven enforcement across Europe\cite{koltay2023censorship}. Knockel et al.\ investigate platform-level censorship on VKontakte, showing how in-platform moderation practices shape the Russian online information environment\cite{knockel2023not}.

\subsection*{Technical and Measurement-Based Studies}
Measurement-driven analyses constitute a large portion of research on wartime information controls. Singh and Acharya study censorship experienced by Ukrainian users, including DNS blocking, access failures, attribution using traceroute, and the potential role of Starlink in circumventing Russian interference\cite{singhcyberspace}. Their findings show limited evidence of Russian-origin blocking and suggest that Starlink does not significantly alter the network-layer censorship landscape.

Ramesh et al.\ conduct a large-scale analysis of network responses immediately following the 2022 invasion, documenting routing changes, filtering mechanisms, outages, and the broader impact on internet freedom\cite{ramesh2023network}. Eichensehr examines cyber operations such as the Viasat satellite attack, limited wiper malware deployments, DDoS campaigns, and the unexpectedly modest role of cyberattacks during the initial invasion\cite{Eichensehr_2022}.

Tavakkoli et al.\ analyze attack patterns across sectors, identifying targeted disruptions against news media, government services, financial systems, and essential consumer platforms\cite{Tavakkoli2025Frontlines}. Their work shows strategic timing of attacks, repeated victimization of high-impact websites, and limited uptake of DDoS protection across affected domains.

\subsection*{Socio-Political Ramifications}
Human-centered and socio-political studies emphasize the consequences of censorship and cyber operations on civilians. Kulyk et al.\ document how network outages, censorship, data breaches, impersonation, and disrupted communication produce cascading harms, including inability to verify safety of family members, loss of access to critical services, exposure of sensitive information, and increased psychological distress\cite{10.1145/3706599.3719906}. Their findings show that seemingly minor disruptions can trigger severe secondary harms, erode public trust, and impede effective decision-making during crises.

Tavakkoli et al.\ further emphasize how strategically timed cyberattacks—especially those targeting media and government services—can shape public perception, destabilize daily life, and create social unrest\cite{Tavakkoli2025Frontlines}.

\subsection*{Summary}
Our work extends this literature by providing a systematic overview that integrates these technical, legal, and socio-political perspectives. Unlike studies that focus narrowly on short-term attacks or isolated mechanisms, we synthesize four years of censorship practices, infrastructure disruptions, and civilian impacts to offer a comprehensive map of information control throughout the ongoing Russia--Ukraine conflict.
