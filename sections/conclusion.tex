\section{Conclusion}\label{sec:title}

Our summary of knowledge has established that censorship in the Russo-Ukrainian war comes from many sources, but impacts Russian citizens the most. During the beginning of the war, Russia carried out attacks on Ukrainian internet infrastructure, but since then has focused more on withdrawing access to its information from the global internet and censoring its own citizens. The Russian government systematically reduces democratic information sources, causing citizens to be isolated from critical thinking and free speech. This also results in populations on both sides of the conflict being less willing to engage with political information and people in uninvolved countries being less aware of the situation as it unfolds.

While many studies on censorship and other forms of internet attacks were done in the year following the invasion in February 2022, the pace of research on this conflict as it develops has slowed down, even as the studies that have been done on more recent developments have shown an increase in censorship. Moreover, current research centers on cross-border cyberattacks, while overlooking domestic censorship as a war tactic. Political and legal studies have realized the implications of Russia’s split from the global internet and the retaliation in the form of internet sanctions taken on by tech companies, as well as educational and government organizations, but minimal technical research has focused on recent developments in implementation. Further research should expand the present state of knowledge around censorship in this war by analyzing the changes in previously observed patterns to determine the areas in which censorship has increased, decreased, or shifted in its implementation, as well as the impacts, especially on Russian citizens.
