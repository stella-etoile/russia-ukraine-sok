\section{Conclusion}\label{sec:conclusion}

The Russo–Ukrainian War demonstrates how modern armed conflict is deeply intertwined with information control, internet censorship, and cyber operations. Across four years of research, scholars have documented technical attacks, platform-level moderation, state-imposed legal restrictions, and the social and psychological harms experienced by civilians. 

This SoK consolidates these findings to provide a comprehensive picture of censorship practices and their consequences. While both Russia and Ukraine have engaged in forms of information control, the motivations, mechanisms, and impacts vary widely. Measurement studies reveal mixed evidence regarding the origin of network interference; legal analyses highlight tension between wartime responses and human-rights standards; and human-centered work shows that even small disruptions can cascade into significant harm.

By synthesizing these perspectives, we aim to create a foundation for future research, support critical evaluation of defensive measures, and contribute to a broader understanding of information warfare as it continues to evolve.
