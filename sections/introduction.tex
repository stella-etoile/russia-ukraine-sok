\section{Introduction}\label{sec:intro}

Russia's full-scale invasion of Ukraine on February 24, 2022 initiated one of the most significant periods of information warfare in recent history. Both Russia and Ukraine, as well as other countries taking sides in the conflict, have engaged in censorship practices, internet restrictions, and information control strategies that affect civilian access to communication services and online content, and the stability of critical services. Russia is the main offender, using censorship techniques to effectively isolate itself from the global internet, but the retaliatory actions of other countries have exacerbated the situation.      
Although many researchers have examined censorship during this conflict, the resulting literature is fragmented. Early work concentrated on attacks and disruptions immediately following the invasion, while the few later studies that exist analyze evolving censorship mechanisms and the broader social and political implications. As the Russo-Ukrainian War continues, a thorough review of the entire body of censorship-related research is needed.
This summary of knowledge (SoK) synthesizes findings across technical measurement studies, socio-political analyses, and legal perspectives. Our objective is to provide a clear and comprehensive understanding of how internet censorship has unfolded throughout the conflict, what mechanisms have been used, what consequences have emerged, and what gaps remain for future work.
