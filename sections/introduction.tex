\section{Introduction}\label{sec:intro}

Russia’s full-scale invasion of Ukraine in February 2022 initiated one of the most significant periods of information warfare in recent history. Both Russia and Ukraine have engaged in censorship practices, internet restrictions, and information control strategies that affect civilian access to communication, the availability of online content, and the stability of critical services. 

Although many researchers have examined censorship during this conflict, the resulting literature is fragmented. Early work concentrated on attacks and disruptions immediately following the invasion, while later studies analyze long-term patterns, evolving censorship mechanisms, and the broader social and political implications. As the Russo–Ukrainian War continues, a systematic appraisal of the entire body of censorship-related research is needed.

This SoK synthesizes findings across technical measurement studies, socio-political analyses, legal perspectives, and examinations of civilian harms. Our objective is to provide a clear and comprehensive understanding of how internet censorship has unfolded throughout the conflict, what mechanisms have been used, what consequences have emerged, and what gaps remain for future work.
